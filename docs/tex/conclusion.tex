\ifdefined\ABSTRACT
\section{Conclusions}
\else
\section{Conclusions and Future Work}
\fi
We developed an efficient and secure distributed key-value store which
can be an answer for a long standing problem: How to authenticate
public keys without relying on central authorities. With BFTKV,
users no longer need to confirm peers' fingerprints or to scan a QR
code in person to ``authenticate'' the public key, which is the ways
most secure chat apps do at the moment. For example, see ``Security
considerations'' from Signal protocols which is one of the most
acclaimed peer-to-peer secure protocols \cite{signal}.
The graph driven architecture gives flexibility to the system, which
makes it possible to join / leave freely without any additional
authorization mechanism. This property makes the system transparent to
anyone, which is especially important for end-to-end systems where we
do not trust any entity even if it is the one who is running the
application systems.

Unlike most other key-value stores, BFTKV provides strong encryption
and signature schemes with a threshold cryptosystem as well. With
these features and the robust and open key-value storage, the system
will be an ideal platform to build blockchain technologies.

Also with the unique threshold password authentication system, it
solves problems like recovering from a key-loss situation and
supporting multiple devices under the same user ID. Most permissioned
systems will suffer these nasty problems when they actually deploy a
system. BFTKV integrates all those features into a Byzantine
fault-tolerant distributed key-value store.

\ifdefined\ABSTRACT
\else
\subsection*{Future Work}
Scalability of bitcoin blockchain is excellent -- you can add as many
nodes as you want and it will not affect the whole system, but the
latency and throughput of the whole process is terrible -- it takes
one hour to settle a transaction.
On the other hand, BFT type of blockchain settles transactions in a
range from sub milliseconds to a few seconds, but scaling out is
difficult. Specifically, with quorum systems we have theoretical lower
bounds such as $n = 3f + 1$ and we cannot do anything about it.

BFTKV can separate the role of the node into two parts: signing and
reading / writing, corresponding to the auth quorum and kv quorum for
load balancing.
Once a transaction gets a collective signature, the transaction
can be stored in any way. The only concern is how to know a
transaction is latest. Each transaction has the timestamp and we can
easily get the latest $\langle x, t, v \rangle$ corresponding to a $x$
in a node but we do not know if other nodes might have a newer
value. BFTKV addresses this issue by a simple quorum system, i.e., as
long as $\forall Q_1, Q_2 \in QS, |Q_1 \cap Q_2| > f$ holds, at least
one node in a quorum has the latest value and the client will choose
that one. This method is simple but it is difficult to scale out. We
need to collect data from at least $f+1$ nodes, which depends on the
total number of servers.

Another concern is the number of collective signatures. Increasing the
number of servers does not help improve the throughput at all, because
every available server in quorum cliques needs to get invovled in
the signing process. Also it will increase the number of the
collective signatures so that each client and kv node needs to verify
more signatures.
One of the ideas to address this issue is to use a threshold signature
to combine the set of collective signatures so that each node verifies
the signature only once. But we need a dealer for the threshold
signature scheme and it will contradict the decentralized concept and
make the system less flexible.
\fi
