\ifdefined\ABSTRACT
\section{Conclusions}
\else
\section{Conclusions and Future Work}
\fi
We developed an efficient and secure distributed key-value store which
can be a solution for a long standing problem: How to authenticate
public keys without relying on central authorities. Here is a quote
from ``The Sesame Algorithm'' \cite{signal} which is a part of the
Signal protocol:
\begin{quote}
``Sesame relies on users to authenticate identity public keys with their
correspondents. For example, a pair of users might compare public key
fingerprints for all their devices manually, or by scanning a QR
code. The details of authentication methods are outside the scope of
this document.

If authentication is not performed, users receive no cryptographic
guarantee as to who they are communicating with.''
\end{quote}
BFTKV is designed to be the last piece of this kind of real end-to-end
encryption system.

We also integrated unique threshold password authentication and
distributed signature schemes with a Byzantine fault-tolerant
key-value store to solve nasty problems like unauthorized enrollment,
recovering from a key-loss situation and supporting multiple devices
under the same ID, which most permissioned blockchain systems will
suffer when they actually try to deploy a system.

\ifdefined\ABSTRACT
\else
\subsection*{Future Work}
Scalability of bitcoin blockchain is excellent -- you can add as many
nodes as you want and it will not affect the whole system, but the
latency and throughput of the whole process is terrible -- it takes
one hour to settle a transaction.
On the other hand, BFT type of blockchain settles transactions in a
range from sub milliseconds to a few seconds, but scaling out is
difficult. Specifically, with quorum systems we have theoretical lower
bounds such as $n = 3f + 1$ and we cannot do anything about it.

BFTKV can separate the role of the node into two parts: signing and
reading / writing, corresponding to the auth quorum and kv quorum for
load balancing.
Once a transaction gets a collective signature, the transaction
can be stored in any way. The only concern is how to know a
transaction is latest. Each transaction has the timestamp and we can
easily get the latest $\langle x, t, v \rangle$ corresponding to a $x$
in a node but we do not know if other nodes might have a newer
value. BFTKV addresses this issue by a simple quorum system, i.e., as
long as $\forall Q_1, Q_2 \in QS, |Q_1 \cap Q_2| > f$ holds, at least
one node in a quorum has the latest value and the client will choose
that one. This method is simple but it is difficult to scale out. We
need to collect data from at least $f+1$ nodes, which depends on the
total number of servers.

Another concern is the number of collective signatures. Increasing the
number of servers does not help improve the throughput at all, because
every available server in quorum cliques needs to get invovled in
the signing process. Also it will increase the number of the
collective signatures so that each client and kv node needs to verify
more signatures.
One of the ideas to address this issue is to use a threshold signature
to combine the set of collective signatures so that each node verifies
the signature only once. But we need a dealer for the threshold
signature scheme and it will contradict the decentralized concept and
make the system less flexible.
\fi
